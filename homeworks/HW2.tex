\documentclass[10pt]{article}
\usepackage[utf8]{inputenc}
\usepackage[T1]{fontenc}
\usepackage{amsmath}
\usepackage{amsfonts}
\usepackage{amssymb}
\usepackage[version=4]{mhchem}
\usepackage{stmaryrd}

\begin{document}
\section*{Math 494 \\
 Homework Set 2}
This assignment should be submitted in Gradescope by Thursday, January 22, at 11:59pm.

Problem 1. Let \(R\) be a ring and \(a, b \in R\) such that \(a b=1\) and \(b a \neq 1\).\\
i) Show that \(1-b a\) is an idempotent element.\\
ii) Show that for every positive integer \(n\), the element \(b^{n}(1-b a)\) is nilpotent (an element \(u \in R\) is nilpotent if \(u^{m}=0\) for some positive integer \(m\) ).\\
iii) Deduce that \(R\) has infinitely many nilpotent elements.

Problem 2. Let \(I\) be an ideal in a commutative ring \(R\).\\
i) The radical of \(I\) is

\[
\operatorname{rad}(I)=\left\{a \in R \mid a^{n} \in I \text { for some } n \in \mathbf{Z}_{>0}\right\}
\]

Show that \(\operatorname{rad}(I)\) is an ideal of \(R\).\\
ii) In particular, deduce that the set of all nilpotent elements of \(R\) is an ideal of \(R\) (this is called the nil-radical of \(R\) ).

Problem 3. Show that if \(R\) is a domain, then \(R[[x]]\) is a domain.\\
Problem 4. Let \(R\) be a commutative ring.\\
i) Show that the polynomial ring \(R[x]\) is never a field.\\
ii) More generally, show that a polynomial \(P=a_{0}+a_{1} x+\ldots+a_{n} x^{n}\) is invertible in \(R[x]\) if and only if \(a_{0}\) is invertible in \(R\) and \(a_{1}, \ldots, a_{n}\) are nilpotent.\\
iii) Show that a polynomial \(P=a_{0}+a_{1} x+\ldots+a_{n} x^{n}\) is nilpotent in \(R[x]\) if and only if \(a_{0}, \ldots, a_{n}\) are nilpotent.

Problem 5. Let \(R\) be a commutative ring and \(f=\sum_{i \geq 0} a_{i} x^{i} \in R[[x]]\). Show that \(f\) is invertible in \(R[[x]]\) if and only if \(a_{0}\) is invertible in \(R\).


\end{document}