\documentclass[10pt]{article}
\usepackage[utf8]{inputenc}
\usepackage[T1]{fontenc}
\usepackage{amsmath}
\usepackage{amsfonts}
\usepackage{amssymb}
\usepackage[version=4]{mhchem}
\usepackage{stmaryrd}

\title{Math 493 \\
 Homework Set 3 }

\author{}
\date{}


\begin{document}
\maketitle
This assignment should be submitted in Gradescope by Thursday, September 25, at 11:59pm.

Problem 1. Let \(n \geq 2\) and consider the symmetric group \(S_{n}\).\\
i) Show that \(S_{n}\) is generated by the permutations ( \(i, i+1\) ), for \(1 \leq i \leq n-1\).\\
ii) Show that \(S_{n}\) is generated by \((1,2)\) and \((1,2, \ldots, n)\).

Problem 2. Let \(n \geq 1\) be an integer. Recall that every permutation \(\sigma \in S_{n}\) can be written as the product of disjoint cycles and the cycles that appear are unique up to reordering.\\
i) Show that if we write \(\sigma=\sigma_{1} \ldots \sigma_{r}\) as a product of disjoint cycles, the for every \(\alpha \in S_{n}\), we have \(\alpha \sigma \alpha^{-1}=\sigma_{1}^{\prime} \cdots \sigma_{r}^{\prime}\), where if \(\sigma_{i}=\left(a_{i 1}, \ldots, a_{i n_{i}}\right)\), then \(\sigma_{i}^{\prime}= \left(\alpha\left(a_{i 1}\right), \ldots, \alpha\left(a_{i n_{i}}\right)\right)\).\\
ii) Deduce that two permutations \(\sigma, \tau \in S_{n}\) are conjugate if and only if they have the same cycle type, that is, when written as a product of disjoint cycles, they have the same number of \(k\)-cycles for all \(k\) (two elements \(x\) and \(y\) in a group \(G\) are conjugate if there is \(g \in G\) such that \(y=g x g^{-1}\) ).

Problem 5. Show that if a permutation \(\sigma \in S_{n}\), when decomposed into disjoint cycles has cycles of length \(d_{1}, \ldots, d_{r}\), then the order of \(\sigma\) is the least common multiple of \(d_{1}, \ldots, d_{r}\).

Problem 4. Let \(n \geq 3\) be an integer. Describe the orders of the elements of the dihedral group \(D_{2 n}\).

The result in the next problem is important to keep in mind. It is known as the Chinese Remainder Theorem.

Problem 5. Let \(m\) and \(n\) be relatively prime positive integers.\\
i) Show that there is a group homomorphism

\[
f: \mathbf{Z} / m n \mathbf{Z} \rightarrow \mathbf{Z} / m \mathbf{Z} \times \mathbf{Z} / n \mathbf{Z}
\]

such that \(f(a+m n \mathbf{Z})=(a+m \mathbf{Z}, a+n \mathbf{Z})\) for all \(a \in \mathbf{Z}\).\\
ii) Show that the above map is injective and deduce that it is a group isomorphism.\\
iii) Show that \(f\) also induces an isomorphism of groups with respect to multiplication

\[
(\mathbf{Z} / m n \mathbf{Z})^{\times} \simeq(\mathbf{Z} / m \mathbf{Z})^{\times} \times(\mathbf{Z} / n \mathbf{Z})^{\times}
\]

Recall that Lagrange's theorem says that if \(G\) is a finite group and \(H\) is a subgroup, then \(|H|\) divides \(|G|\). The next problem gives an example of a finite group and some \(d\) dividing \(|G|\) such that \(G\) has no subgroup of order \(d\).

Problem 6. Let \(G=A_{4}\), so \(|G|=12\). Show that \(G\) has no subgroup of order 6, as follows:\\
i) Suppose that \(H\) is a subgroup of \(G\) of order 6. Show that for every \(g \in G\), we have \(g^{2} \in H\). Hint: consider what \(g^{2} H\) might be.\\
ii) Deduce that every 3 -cycle lies in \(H\).\\
iii) By counting the number of 3 -cycles in \(A_{4}\), deduce a contradiction.


\end{document}